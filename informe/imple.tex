\section{Implementación en RDBMS}
A partir del modelo relacional presentado, realizamos la implementación del mismo para MariaDB 10.1.

Se entregan los siguientes archivos:

\begin{itemize}
    \item SQL\_creacion\_tablas.sql \\
    Permite generar la estructura de la base presentada en el Modelo Relacional
    \item SQL\_populacion\_tablas.sql \\
    Permite rellenar dicha estructura con datos de prueba e implementa las consultas solicitadas por la cátedra.
\end{itemize}

\section{Funcionalidades}

\begin{itemize}
  \item \textbf{Estadísticas: atracción que más facturó, parque que más facturó, atracción que más facturó por parque.} \\
  SP: \textit{sp1()}
  \item \textbf{Listado de facturas adeudadas} \\
  SP: \textit{facturasAdeudadas()}
  \item \textbf{Para cada cliente las atracciones más visitadas en rango de fechas} \\
  SP: \textit{atraccionesPorCliente(desde, hasta)} \\
  Ej: \textit{atraccionesPorCliente('2017-01-30', '2017-05-20')}
  \item \textbf{Cambios de categorías de cliente en rango de fechas} \\
  SP: \textit{cambiosCategoriaCliente(dni, desde, hasta)} \\
  Ej: \textit{cambiosCategoriaCliente(13241345, '2016-12-30', '2017-04-01')}
  \item \textbf{Atracciones con descuento para cada categoría.} \\
  SP: \textit{atraccionesDescuentoPorCategoria()}
  \item \textbf{Empresa organizadora de eventos que tuvo mayor facturación.} \\
  SP: \textit{empresaMasFacturo()}
  \item \textbf{Desarrollar un procedimiento almacenado que verifique las categorías y realice el cambio de la misma si es necesario.} \\
  SP: \textit{sp7()}
  \item \textbf{Ranking de parques/atracciones con mayor cantidad de visitas en rango de fechas.} \\
  SP: \textit{rankingParquesAtracciones(desde, hasta)} \\
  Ej: \textit{rankingParquesAtracciones('2017-01-30', '2017-05-20')} \\
\end{itemize}

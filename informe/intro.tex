\section{Introducción}
En el presente trabajo práctico se nos solicitó que diseñaramos la base de datos de un sistema de control de entradas. El mismo debería poder soportar múltiples empresas, múltiples locaciones, ya sean eventos temporarios o parques permanentes, además del control de acceso a las atracciones, descuentos para los socios y etc.
Una vez modelado el mismo mediante el diagrama entidad relación, se pasa a realizar el modelo relacional asociado al mismo para finalmente realizar la implementación en un motor de bases de datos.

Dentro del problema se pueden identificar diversas entidades principales,
entre ellas encontramos: 

\begin{itemize}
\item Eventos
\item Parques
\item Atracciones
\item Clientes
\item Tarjetas
\item Empresas
\item Modos de pago
\item Facturas
\item Equipos
\end{itemize}


Presentamos a continuación, el modelo de entidad-relación que los incluye para representar en sistema descripto y el modelo relacional derivado del mismo. Aparte, entregamos la solución en la forma de una implementación en un motor de base de datos, populado con datos de ejemplo para poder ejecutar las diversas \textit{queries} que resuelven el problema.

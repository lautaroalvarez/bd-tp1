\section{Modelo Relacional Derivado}
A partir del MER realizado, derivamos un Modelo Relacional, para pasar 
nuestro modelo a un formato de tablas que luego nos servirá para implementar 
la solución. \\
\\
\textbf{locacion}(\underline{idLocacion}, nombre, ubicacion, precio,tipo) \\
\textbf{empresa}(\underline{cuitEmpresa}, razonSocial, direccion, provincia, pais) \\
\textbf{evento}(\underline{idLocacion}, \dashuline{cuitEmpresa}, fechaInicio, fechaFin) \\
\textbf{atraccion}(\underline{idAtraccion}, \dashuline{idLocacion}, nombre, precio, minimoEdad, minimoAltura) \\
\textbf{categoria}(\underline{idCategoria}, nombre, valorX, valorY) \\
\textbf{descuentoEnLocacion}(\underline{idCategoria, idLocacion}, porcentaje) \\
\textbf{descuentoEnAtraccion}(\underline{idCategoria, idAtraccion}, porcentaje) \\
\textbf{modoDePago}(\underline{idModoDePago}, nombre) \\
\textbf{cliente}(\underline{dni}, nombre, apellido, direccion, telefono, \dashuline{idModoDePago}, fechaAlta, \dashuline{idCategoria}) \\
\textbf{clienteTuvoCategoria}(\underline{dni, idCategoria, fechaDesde}) \\
\textbf{tarjeta}(\underline{numeroTarjeta}, \dashuline{dni}, foto, estado, fechaAlta) \\
\textbf{clienteTuvoTarjeta}(\underline{numeroTarjeta}, \dashuline{dni}, fechaInicio) \\
\textbf{factura}(\underline{numeroFactura}, \dashuline{dni}, fecha, monto, estado, fechaVencimiento) \\
\textbf{entrada}(\underline{idEntrada}, \dashuline{dni}, \dashuline{numeroTarjeta}, fecha, precio, nroDeFactura, \dashuline{numeroFactura}, fecha, precio) \\
\textbf{entradaALocacion}(\underline{idEntrada}, \dashuline{idLocacion}) \\
\textbf{entradaAAtraccion}(\underline{idEntrada}, \dashuline{idAtraccion}) \\
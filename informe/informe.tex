\documentclass[a4paper,10pt]{article}

\usepackage[margin=1in]{geometry} 	% Setea el margen manualmente, todos iguales.
\usepackage[spanish]{babel} 		% {Con estos dos anda
\usepackage[utf8]{inputenc} 		% todo lo que es tildes y ñ}
\usepackage{fancyhdr} 			%{Estos dos son para
\pagestyle{fancyplain} 			% el header copado}
\usepackage{color}			% Con esto puedo hacer la matufia de poner en color blanco un texto para engañar al formato
\usepackage{ulem}
\usepackage{caratula/caratula}
\usepackage{hyperref}
\usepackage{pdfpages}
\usepackage{graphicx}


\lhead{Bases de Datos} 	% {Con esto se usa el header copado. También está \chead para
\rhead{Trabajo Práctico 1 - Grupo 13} 	% el centro y comandos para el pie de página, buscar fancyhdr}

%%%%%%%
% Macros  %
%%%%%%%

% Tupla para hacer MR
% Uso: \tuple{NombreTabla}{Contenido de la tupla separado por comas}
% Pone en negrita el nombre de la tabla y envuelve en paréntesis el contenido. Recomiendo usar junto con \pk{} y \fk{} de ser necesario.
\newcommand{\tuple}[2]{\textbf{#1}(#2) \\}
% Primary key
% Uso: \pk{nombreDePK}
% Subraya la key.
\newcommand{\pk}[1]{\underline{#1}}
% Foreign key
% Uso: \fk{nombreDeFK}
% Subraya con línea punteada la key.
\newcommand{\fk}[1]{\dashuline{#1}}
% Restricción adicional
% Uso: \ra{Contenido de la restricción}
% Escribe "RA:" seguido de la restricción. Recomiendo usar junto con \attr{}.
\newcommand{\ra}[1]{RA: #1 \\}
% Atributo
% Uso: \attr{nombreDeAtributo}
% Pone al atributo en itálica.
\newcommand{\attr}[1]{\textit{#1}}
% Nota
% Uso: \nota{Contenido de la nota}
% Escribe "Nota:" seguido de la nota. Recomiendo usar junto con \attr{}.
\newcommand{\nota}[1]{\textit{Nota:} #1 \\}



\begin{document}

%Datos para la caratula
\fecha{\today}
\materia{Base de Datos}
\titulo{Trabajo Práctico I}

\integrante{Lautaro Leonel Alvarez}{268/14}{lautarolalvarez@gmail.com}
\integrante{Ignacio Rodriguez}{797/13}{igna.r286@gmail.com}
\integrante{Diego Sueiro}{75/90}{dsueiro@gmail.com}
\integrante{Pablo Somodi}{818/10}{pablo@somodi.com.ar}

\maketitle

\tableofcontents

\newpage

\section{Introducción}
En el presente trabajo práctico se nos solicitó que diseñaramos la base de datos de un sistema de control de entradas. El mismo debería poder soportar múltiples empresas, múltiples locaciones, ya sean eventos temporarios o parques permanentes, además del control de acceso a las atracciones, descuentos para los socios y etc.
Una vez modelado el mismo mediante el diagrama entidad relación, se pasa a realizar el modelo relacional asociado al mismo para finalmente realizar la implementación en un motor de bases de datos.

Dentro del problema se pueden identificar diversas entidades principales,
entre ellas encontramos: 

\begin{itemize}
\item Eventos
\item Parques
\item Atracciones
\item Clientes
\item Tarjetas
\item Empresas
\item Modos de pago
\item Facturas
\item Equipos
\end{itemize}


Presentamos a continuación, el modelo de entidad-relación que los incluye para representar en sistema descripto y el modelo relacional derivado del mismo. Aparte, entregamos la solución en la forma de una implementación en un motor de base de datos, populado con datos de ejemplo para poder ejecutar las diversas \textit{queries} que resuelven el problema.


\section{Modelo de Entidad Relación}

En la siguiente página se puede observar el diagrama de entidad relación.


\includepdf[pages={-},fitpaper,rotateoversize]{der.pdf}


\section{Modelo Relacional Derivado}
A partir del MER realizado, derivamos un Modelo Relacional, para pasar 
nuestro modelo a un formato de tablas que luego nos servirá para implementar 
la solución. \\
\\
\textbf{locacion}(\underline{idLocacion}, nombre, ubicacion, precio,tipo) \\
\textbf{empresa}(\underline{cuitEmpresa}, razonSocial, direccion, provincia, pais) \\
\textbf{evento}(\underline{idLocacion}, \dashuline{cuitEmpresa}, fechaInicio, fechaFin) \\
\textbf{atraccion}(\underline{idAtraccion}, \dashuline{idLocacion}, nombre, precio, minimoEdad, minimoAltura) \\
\textbf{categoria}(\underline{idCategoria}, nombre, valorX, valorY) \\
\textbf{descuentoEnLocacion}(\underline{idCategoria, idLocacion}, porcentaje) \\
\textbf{descuentoEnAtraccion}(\underline{idCategoria, idAtraccion}, porcentaje) \\
\textbf{modoDePago}(\underline{idModoDePago}, nombre) \\
\textbf{cliente}(\underline{dni}, nombre, apellido, direccion, telefono, \dashuline{idModoDePago}, fechaAlta, \dashuline{idCategoria}) \\
\textbf{clienteTuvoCategoria}(\underline{dni, idCategoria, fechaDesde}) \\
\textbf{tarjeta}(\underline{numeroTarjeta}, \dashuline{dni}, foto, estado, fechaAlta) \\
\textbf{clienteTuvoTarjeta}(\underline{numeroTarjeta}, \dashuline{dni}, fechaInicio) \\
\textbf{factura}(\underline{numeroFactura}, \dashuline{dni}, fecha, monto, estado, fechaVencimiento) \\
\textbf{entrada}(\underline{idEntrada}, \dashuline{dni}, \dashuline{numeroTarjeta}, fecha, precio, nroDeFactura, \dashuline{numeroFactura}, fecha, precio) \\
\textbf{entradaALocacion}(\underline{idEntrada}, \dashuline{idLocacion}) \\
\textbf{entradaAAtraccion}(\underline{idEntrada}, \dashuline{idAtraccion}) \\

\newpage
\section{Implementación en RDBMS}
A partir del modelo relacional presentado, realizamos la implementación del mismo para MariaDB 10.1.

Se entregan los siguientes archivos:

\begin{itemize}
    \item SQL\_creacion\_tablas.sql \\
    Permite generar la estructura de la base presentada en el Modelo Relacional
    \item SQL\_populacion\_tablas.sql \\
    Permite rellenar dicha estructura con datos de prueba e implementa las consultas solicitadas por la cátedra.
\end{itemize}

\section{Funcionalidades}

\begin{itemize}
  \item \textbf{Estadísticas: atracción que más facturó, parque que más facturó, atracción que más facturó por parque.} \\
  SP: \textit{sp1()}
  \item \textbf{Listado de facturas adeudadas} \\
  SP: \textit{facturasAdeudadas()}
  \item \textbf{Para cada cliente las atracciones más visitadas en rango de fechas} \\
  SP: \textit{atraccionesPorCliente(desde, hasta)} \\
  Ej: \textit{atraccionesPorCliente('2017-01-30', '2017-05-20')}
  \item \textbf{Cambios de categorías de cliente en rango de fechas} \\
  SP: \textit{cambiosCategoriaCliente(dni, desde, hasta)} \\
  Ej: \textit{cambiosCategoriaCliente(13241345, '2016-12-30', '2017-04-01')}
  \item \textbf{Atracciones con descuento para cada categoría.} \\
  SP: \textit{atraccionesDescuentoPorCategoria()}
  \item \textbf{Empresa organizadora de eventos que tuvo mayor facturación.} \\
  SP: \textit{empresaMasFacturo()}
  \item \textbf{Desarrollar un procedimiento almacenado que verifique las categorías y realice el cambio de la misma si es necesario.} \\
  SP: \textit{sp7()}
  \item \textbf{Ranking de parques/atracciones con mayor cantidad de visitas en rango de fechas.} \\
  SP: \textit{rankingParquesAtracciones(desde, hasta)} \\
  Ej: \textit{rankingParquesAtracciones('2017-01-30', '2017-05-20')} \\
\end{itemize}


\section{Conclusiones}

La elaboración del trabajo nos permitió observar las diferentes alternativas que nos dan los diagramas de entidad relación para modelar una misma situación.

Las discusiones, por ejemplo, sobre si usar una relación ternaria o una entidad para un mismo concepto o cuán literal había que tomar lo indicado por ``nuestro cliente'' nos hicieron observar desde las virtudes expresivas del lenguaje hasta su vinculación con el futuro modelo relacional a implementar.



\end{document}
